% ----------------------------------------------------------------
% Article Class (This is a LaTeX2e document)  ********************
% ----------------------------------------------------------------
\documentclass[12pt]{article}
\usepackage[english]{babel}
\usepackage{amsmath,amsthm}
\usepackage{amsfonts}
\usepackage{geometry}
\usepackage{enumerate}
\usepackage{graphicx}
\geometry{left=2.5cm,right=2.5cm,top=2cm,bottom=2cm}
% THEOREMS -------------------------------------------------------
\newtheorem{thm}{Theorem}[section]
\newtheorem{cor}[thm]{Corollary}
\newtheorem{lem}[thm]{Lemma}
\newtheorem{prop}[thm]{Proposition}
\theoremstyle{definition}
\newtheorem{defn}[thm]{Definition}
\theoremstyle{remark}
\newtheorem{rem}[thm]{Remark}
\numberwithin{equation}{section}
% ----------------------------------------------------------------
\begin{document}
%Title------------------------------------------------------------
\title{\textsc{CSE221: Assignment 0}}%
\date{}
\author{\small Qiheng \textsc{Wang}}%
\maketitle
\noindent{\scriptsize Assignment 1: due Wednesday, April 3, 2013, in class.\\

%Question1------------------------------------------------------------
	1. Privileged instructions:
	A privileged instruction is one that is only allowed to be executed in
	Ring-0
	(i.e. kernel mode).
	a. Set value of timer: privileged
	Only the kernel should be cable to set value of the timer, otherwise user
	processes could dominate CPU usage. This may lead to unintended results such
	as preventing other processes from accessing resource.

	b. Read the clock: unprivileged
	User processes can't really harm the system by merely reading the clock. On
	the other hand If all processes need to make a system call everytime they need
	to read the clock, the expense would be too huge.

	c. Clear memory: privileged
	User processes should not be allowed to clear memory of any other processes.

	d. Turn off interrupts: privileged
	The kernel handles interrupts. If interrupts can be turned of by a user
	processes, the kernel may never gains control.

	e. Switch from user to kernel mode: unprivileged
	User processes should be able to enter kernel mode by making system calls,
	which themselves don't need to be privileged.

%Question2------------------------------------------------------------
	2.1
	class CountdownEvent {

		private int counter;
		private bool isSignalled; // 0: nonsignalled, 1: signalled
		private Lock lock; // pthread_mutex_lock
		private Condition cond; //pthread_cond_broadcast

		CountdownEvent (int count) {
			this.counter = count;
			isSignalled = count > 0 ? false : true;
			lock = new Lock();
			cond = new Condition();
		}
		void Increment() {
			lock.Acquire();
			if (isSignalled) return;
			this.counter++;
			lock.Release();
		}
		void Decrement() {
			lock.Acquire();
			if (isSignalled) return;
			this.counter--;
			if (this.counter == 0) {
				isSignalled = true;
				cond.Broadcast(); // unblock ALL wait list thread
			}
			lock.Release();
		}
		void Wait() {
			lock.Acquire();
			if (isSignalled) return;
			cond.Wait(&lock); //wait on the condition variable mutex
			lock.Release();
		}
	}

	2.2 Semantic difference between CountdownEvent and Semaphore:
	CountdownEvent can be regarded as an inverse Semaphore.
	In the case of Microsoft .NET:
	Semaphore: Threads call WaitOne() before running, which decrements the
	semaphore, and call Release() after done, which increments it.
	CountdownEvent: Threads call Increment() before running, which increments
	the counter, and call Decrement() after done, which decrements it. Then

	There's some other implied differences:
	Block state: Threads always explicitly block on CountdownEvent when calling Wait().
	On the other hand, threads may or may not block on Semaphores. (It depends
	on the internal state of the semaphore counter.)
	Initial value: Semaphore can be zero (a private samaphore for order synchronization),
	CountdownEvent cannot, since it would be set to "signalled" and return.
	Re-usability: CountdownEvent is a single usage object, once hit zero, it
	does nothing other than return. Semaphore is reusable. the value can go
	back and forth.
	Value: For n parallel processes, Semaphore's value varies from 1-n to 1.
	CountdownEvent's value varies from 0 to n
	Wake-up: CountdownEvent wakes up all waiting threads, Semaphore only wakes
	up one.

	2.3 Brrier class:
		class Barrier {
			CountdownEvent ev;
			Barrier (int n) {
				ev = new CountdownEvent(n);
			}
			void Done () {
				ce.Decrement();
				ce.Wait();
			}
		}
	Barrier class allows you to parallelize tasks in a sense that a task
	does not continue until all entries in the barrier have signaled their
	arrival. CountdownEvent is a single usage object (need reallocation
	for a new allocation), while Barrier is reusable.

%Question3------------------------------------------------------------
	3 Virtual memory and TLB:
	Assume we have a hardware cache TLB, residing in a chip's MMU.
	Since PTE for P is not in the TLB, this causes a TLB miss. The hardware will
	have to consult the page table in the main memory, via a page-table base
	register (PTBR).  After finding the correct the PTE, update the TLB and retry the memory reference.
	However, as in this case. P is paged out to disk, which means P still can'be found in the main memory, (the
	valid bit being off). This renders a page fault, generating a trap into
	the kernel.
	Once found, read P into a free frame onto the memory. This may need a page
	replacement if there's no room for P.
	Record frame number in the page table entry. Finnaly set valid bit to 1.
	Finally when the instruction is re-executed, the TLB lookup would result in a hit.

	Assume we have a software-managed TLB. The hardware doesn't have to do
	much on a miss as above. It raises an exception, and the OS TLB miss
	handler takes over. More over, the OS can use any data structure it wants
	to implement the page table.
	The advantages of software-managed TLB over hardware is simplicity and
	flexibility.

\end{document}
